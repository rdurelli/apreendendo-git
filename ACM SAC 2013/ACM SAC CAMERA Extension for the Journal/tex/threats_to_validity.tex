\textbf{Primary studies selection}. Aiming at ensuring an unbiased selection process, we defined research questions in advance and devised inclusion and exclusion criteria we believe are detailed enough to provide an assessment of how the final set of primary studies was obtained. However, we cannot rule out threats from a quality assessment perspective, we simply selected studies without assigning any scores. In addition, we wanted to be as inclusive as possible, thus no limits were placed on date of publication and we avoided imposing many restrictions on primary study selection since we wanted a broad overview of the research area.

\textbf{Missing important primary studies}. The search for primary studies was conducted in several search engines, even though it is rather possible we have missed some primary studies. Nevertheless, this threat was mitigated by selecting search engines which have been regarded as the most relevant scientific sources~\cite{Dyba} and therefore prone to contain the majority of the important studies.

\textbf{Reviewers reliability}. All the reviewers of this study are researchers in the software reuse field, focused on the aspect-oriented programming, software testing and software product line, and none of the techniques and tools developed by us. Therefore, we are not aware of any bias we may have introduced during the analyses.

\textbf{Data extraction}. Another threat for this review refers to how the data were extracted from the digital libraries, since not all the information was obvious to answer the questions and some data had to be interpreted. Therefore, in order to ensure the validity, multiple sources of data were analyzed, i.e. papers, technical reports, white papers. Furthermore, in the event of a disagreement between the two primary reviewers, a third reviewer acted as an arbitrator to ensure full agreement was reached.